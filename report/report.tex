\documentclass[twocolumn]{article}
\usepackage[utf8]{inputenc}
\usepackage[style=numeric]{biblatex}
\usepackage{hyperref}
\usepackage{graphicx}
\usepackage{microtype}

% Bibliography file
\addbibresource{references.bib}

\title{Your Paper Title}
\author{Your Name}
\date{\today}

\begin{document}

\maketitle

\begin{abstract}
Your abstract goes here.
\end{abstract}

\section{Introduction}
As deep neural networks continue to dominate the landscape of machine learning
applications, the design of domain-specific hardware accelerators has become
increasingly critical for efficient deployment in deep learning situations.
While these specialized architectures offer significant performance improvements
over general-purpose processors, their design space is heavily influenced by
fundamental assumptions about numerical precision and data representation. One
such assumption -- the bit width used for neural network operations -- can
dramatically impact both the hardware complexity and the achievable performance
of these accelerators.

Model quantization, the process of reducing numerical precision of network
parameters and computations, represents a crucial bridge between neural network
algorithms and their hardware implementations. While current domain-specific
architectures often implement fixed bit widths (typically 8-bit integers or
16-bit floating point numbers) based on conventional wisdom or hardware
constraints, there has been limited systematic investigation into how different
neural architectures respond to varying levels of quantization precision. This
gap in understanding affects not only software optimization but also hardware
design decisions, as the choice of supported bit widths directly impacts circuit
complexity, memory requirements, and energy efficiency of neural network
accelerators.

The motivation for quantization stems from both practical and theoretical
considerations in neural network deployment. From a hardware perspective,
reduced precision operations offer substantial benefits: lower memory bandwidth
requirements, decreased power consumption, and simpler arithmetic units that can
be parallelized more effectively. For example, replacing 32-bit floating-point
operations with 8-bit integer operations can theoretically yield a 4x reduction
in memory footprint and bandwidth requirements. This efficiency gain is
particularly crucial for edge devices and mobile applications where power and
memory constraints are significant limiting factors. Additionally, many
domain-specific neural network accelerators can achieve higher throughput with
fixed-point or reduced-precision floating-point arithmetic, as these simpler
operations allow for more processing units within the same silicon area and
power envelope.

Beyond hardware efficiency, quantization addresses a fundamental question about
neural network representations: how much numerical precision is actually
necessary for effective inference? Neural networks are known to be robust to
various forms of noise and perturbation during training, suggesting they might
not require the full precision of 32-bit floating-point numbers to maintain
their functional capabilities. Understanding this precision requirement is
crucial for both software optimization and hardware design, as it directly
influences decisions about arithmetic unit complexity, memory hierarchy
organization, and data movement strategies in domain-specific accelerators.

This work presents a comprehensive study examining the relationship between
quantization bit depth and model performance across diverse neural network
architectures. Through implementation of a custom quantization framework that
enables exploration of arbitrary bit depths, the study evaluates four distinct
neural architectures: a simple dense network, two different MNIST classifiers
(dense and convolutional), and a transformer-based sequence model.

In summary, my work contributes the following:
\begin{enumerate}
\item A flexible quantization framework for exploring arbitrary bit depths
\item Empirical evaluation of quantization effects across multiple network architectures
\item Analysis of the relationship between bit depth and model performance, introducing a ``signal-to-complexity'' metric
\item Practical insights for informing both software quantization strategies and hardware design decisions
\end{enumerate}

% Your content sections go here

\printbibliography

\end{document}